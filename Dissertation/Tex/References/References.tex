\chapter*{References}
\addcontentsline{toc}{chapter}{References}
\parindent=-2em
\small

Alabed YZ, Pool M, Tone SO, Fournier AE (2007) Identification of CRMP4 as a Convergent Regulator of Axon Outgrowth Inhibition. Journal of Neuroscience 27:1702-1711.

Alabed YZ, Pool M, Ong Tone S, Sutherland C, Fournier AE (2010) GSK3 beta regulates myelin-dependent axon outgrowth inhibition through CRMP4. J Neurosci 30:5635-5643.

Antonini A, Stryker M (1993) Development of individual geniculocortical arbors in cat striate cortex and effects of binocular impulse blockade. The Journal of Neuroscience 13:3549-3573.

Antonini A, Stryker M (1996) Plasticity of geniculocortical afferents following brief or prolonged monocular occlusion in the cat. J Comp Neurol 369:64-82.

Antonini A, Gillespie D, Crair M, Stryker M (1998) Morphology of single geniculocortical afferents and functional recovery of the visual cortex after reverse monocular deprivation in the kitten. J Neurosci 18:9896-9909.

Augustin I, Betz A, Herrmann C, Jo T, Brose N (1999) Differential expression of two novel Munc13 proteins in rat brain. Biochem J 337 ( Pt 3):363-371.

Bal T, von Krosigk M, McCormick D (1995) Synaptic and membrane mechanisms underlying synchronized oscillations in the ferret lateral geniculate nucleus in vitro. J Physiol (Lond) 483:641-663.

Bishop KM, Goudreau G, O'Leary DD (2000) Regulation of area identity in the mammalian neocortex by Emx2 and Pax6. Science 288:344-349.

Bishop KM, Rubenstein JL, O'Leary DD (2002) Distinct actions of Emx1, Emx2, and Pax6 in regulating the specification of areas in the developing neocortex. J Neurosci 22:7627-7638.

Bishop KM, Garel S, Nakagawa Y, Rubenstein JL, O'Leary DD (2003) Emx1 and Emx2 cooperate to regulate cortical size, lamination, neuronal differentiation, development of cortical efferents, and thalamocortical pathfinding. J Comp Neurol 457:345-360.

Blumenfeld H, McCormick D (2000) Corticothalamic inputs control the pattern of activity generated in thalamocortical networks. J Neurosci 20:5153-5162.

Bonhoeffer T, Grinvald A (1991) Iso-orientation domains in cat visual cortex are arranged in pinwheel-like patterns. Nature 353:429-431.

Bonhoeffer T, Grinvald A (1996) Optical Imaging Based on Intrinsic Signals: The Methodology. In: Brain Mapping: The Methods, pp 55-97.

Bosking W, Zhang Y, Schofield B, Fitzpatrick D (1997) Orientation selectivity and the arrangement of horizontal connections in tree shrew striate cortex. J Neurosci 17:2112-2127.

Brittain J, Piekarz A, Wang Y, Kondo T, Cummins T, Khanna R (2009) An atypical role for collapsin response mediator protein 2 (CRMP-2) in neurotransmitter release via interaction with presynaptic voltage-gated Ca2+ channels. J Biol Chem.

Brunso-Bechtold J, Casagrande V (1981) Effect of bilateral enucleation on the development of layers in the dorsal lateral geniculate nucleus. Neuroscience 6:2579-2586.

Bulfone A, Martinez S, Marigo V, Campanella M, Basile A, Quaderi N, Gattuso C, Rubenstein JL, Ballabio A (1999) Expression pattern of the Tbr2 (Eomesodermin) gene during mouse and chick brain development. Mech Dev 84:133-138.
Butts D (2002) Retinal waves: implications for synaptic learning rules during development. Neuroscientist 8:243-253.

Butts D, Feller M, Shatz C, Rokhsar D (1999) Retinal waves are governed by collective network properties. J Neurosci 19:3580-3593.

Cabelli R, Hohn A, Shatz C (1995) Inhibition of ocular dominance column formation by infusion of NT-4/5 or BDNF. Science 267:1662-1666.

Cabelli R, Shelton D, Segal R, Shatz C (1997) Blockade of endogenous ligands of trkB inhibits formation of ocular dominance columns. Neuron 19:63-76.

Cang J, Wang L, Stryker MP, Feldheim DA (2008a) Roles of ephrin-as and structured activity in the development of functional maps in the superior colliculus. J Neurosci 28:11015-11023.

Cang J, Kaneko M, Yamada J, Woods G, Stryker M, Feldheim D (2005) Ephrin-as guide the formation of functional maps in the visual cortex. Neuron 48:577-589.

Cang J, Niell CM, Liu X, Pfeiffenberger C, Feldheim DA, Stryker MP (2008b) Selective disruption of one Cartesian axis of cortical maps and receptive fields by deficiency in ephrin-As and structured activity. Neuron 57:511-523.

Catalano SM, Shatz CJ (1998) Activity-Dependent Cortical Target Selection by Thalamic Axons. Science 281:559-562.

Catterall WA, Few AP (2008) Calcium channel regulation and presynaptic plasticity. Neuron 59:882-901.

Chapman B, Zahs K, Stryker M (1991) Relation of cortical cell orientation selectivity to alignment of receptive fields of the geniculocortical afferents that arborize within a single orientation column in ferret visual cortex. J Neurosci 11:1347-1358.

Chappert-Piquemal C, Fonta C, Malecaze F, Imbert M (2001) Ocular dominance columns in the adult New World Monkey Callithrix jacchus. Vis Neurosci 18:407-412.

Chen MS, Huber AB, van der Haar ME, Frank M, Schnell L, Spillmann AA, Christ F, Schwab ME (2000) Nogo-A is a myelin-associated neurite outgrowth inhibitor and an antigen for monoclonal antibody IN-1. Nature 403:434-439.

Chiu C, Weliky M (2001) Spontaneous Activity in Developing Ferret Visual Cortex In Vivo. The Journal of Neuroscience 21:8906-8914.

Chiu C, Weliky M (2002) Relationship of correlated spontaneous activity to functional ocular dominance columns in the developing visual cortex. Neuron 35:1123-1134.

Cnops L, Van de Plas B, Arckens L (2004) Age-dependent expression of collapsin response mediator proteins (CRMPs) in cat visual cortex. Eur J Neurosci 19:2345-2351.

Cnops L, Hu T-T, Burnat K, Arckens L (2007a) Influence of Binocular Competition on the Expression Profiles of CRMP2, CRMP4, Dyn I, and Syt I in Developing Cat Visual Cortex. Cerebral Cortex 18:1221-1231.

Cnops L, Hu T-T, Eysel UT, Arckens L (2007b) Effect of binocular retinal lesions on CRMP2 and CRMP4 but not Dyn I and Syt I expression in adult cat area 17. Eur J Neurosci 25:1395-1401.

Cnops L, Hu T-T, Burnat K, Van Der Gucht E, Arckens L (2006) Age-dependent alterations in CRMP2 and CRMP4 protein expression profiles in cat visual cortex. Brain Research 1088:109-119.

Cnops L, Hu T-T, Vanden Broeck J, Burnat K, Van Den Bergh G, Arckens L (2007c) Age- and experience-dependent expression of Dynamin I and Synaptotagmin I in cat visual system. J Comp Neurol 504:254-264.

Corriveau R, Huh G, Shatz C (1998) Regulation of class I MHC gene expression in the developing and mature CNS by neural activity. Neuron 21:505-520.

Cramer K, Sur M (1997) Blockade Of Afferent Impulse Activity Disrupts On/Off Sublamination In the Ferret Lateral Geniculate Nucleus. Brain Research Developmental Brain Research 98:287-290.

Crowley J, Katz L (1999) Development of ocular dominance columns in the absence of retinal input. Nat Neurosci 2:1125-1130.

Crowley J, Katz L (2000) Early development of ocular dominance columns. Science 290:1321-1324.

Crowley J, Katz L (2002) Ocular dominance development revisited. Curr Opin Neurobiol 12:104-109.

Cucchiaro J, Guillery R (1984) The development of the retinogeniculate pathways in normal and albino ferrets. Proc R Soc Lond B Biol Sci 223:141-164.

Dergham P, Ellezam B, Essagian C, Avedissian H, Lubell WD, McKerracher L (2002) Rho signaling pathway targeted to promote spinal cord repair. J Neurosci 22:6570-6577.

Drescher U, Kremoser C, Handwerker C, L�schinger J, Noda M, Bonhoeffer F (1995) In vitro guidance of retinal ganglion cell axons by RAGS, a 25 kDa tectal protein related to ligands for Eph receptor tyrosine kinases. Cell 82:359-370.

Dufour A, Seibt J, Passante L, Depaepe V, Ciossek T, Frisen J, Kullander K, Flanagan J, Polleux F, Vanderhaeghen P (2003) Area specificity and topography of thalamocortical projections are controlled by ephrin/Eph genes. Neuron 39:453-465.

Ellsworth CA, Lyckman AW, Feldheim DA, Flanagan JG, Sur M (2005) Ephrin-A2 and -A5 influence patterning of normal and novel retinal projections to the thalamus: Conserved mapping mechanisms in visual and auditory thalamic targets. J Comp Neurol 488:140-151.

Famiglietti E, Kolb H (1976) Structural basis for ON-and OFF-center responses in retinal ganglion cells. Science 194:193-195.

Feller M, Wellis D, Stellwagen D, Werblin F, Shatz C (1996) Requirement for cholinergic synaptic transmission in the propagation of spontaneous retinal waves. Science 272:1182-1187.

Feller MB (1999) Spontaneous Correlated Activity in Developing Neural Circuits. Neuron 22:653-656.

Feller MB, Butts DA, Aaron HL, Rokhsar DS, Shatz CJ (1997) Dynamic Processes Shape Spatiotemporal Properties of Retinal Waves. Neuron 19:293-306.

Flanagan J, Vanderhaeghen P (1998) The ephrins and Eph receptors in neural development. Annu Rev Neurosci 21:309-345.

Frisen J, Yates P, McLaughlin T, Friedman G, O'Leary D, Barbacid M (1998) Ephrin-A5 (AL-1/RAGS) is essential for proper retinal axon guidance and topographic mapping in the mammalian visual system. Neuron 20:235-243.

Frostig R, Lieke E, Ts'o D, Grinvald A (1990) Cortical functional architecture and local coupling between neuronal activity and the microcirculation revealed by in vivo high-resolution optical imaging of intrinsic signals. Proc Natl Acad Sci U S A 87:6082-6086.

Fukata Y, Itoh TJ, Kimura T, M�nager C, Nishimura T, Shiromizu T, Watanabe H, Inagaki N, Iwamatsu A, Hotani H, Kaibuchi K (2002) CRMP-2 binds to tubulin heterodimers to promote microtubule assembly. Nat Cell Biol 4:583-591.

Galli L, Maffei L (1988) Spontaneous impulse activity of rat retinal ganglion cells in prenatal life. Science 242:90-91.

Gallyas F (1979) Silver staining of myelin by means of physical development. Neurol Res 1:203-209.
Ghosh A, Shatz C (1992a) Involvement of subplate neurons in the formation of ocular dominance columns. Science 255:1441-1443.

Ghosh A, Shatz C (1992b) Pathfinding and target selection by developing geniculocortical axons. J Neurosci 12:39-55.

Goddard C, Butts D, Shatz C (2007) Regulation of CNS synapses by neuronal MHC class I. Proc Natl Acad Sci U S A 104:6828-6833.

Godement P, Salaun J, Imbert M (1984) Prenatal and postnatal development of retinogeniculate and retinocollicular projections in the mouse. J Comp Neurol 230:552-575.

Gong L, Puri M, Unlu M, Young M, Robertson K, Viswanathan S, Krishnaswamy A, Dowd S, Minden J (2004) Drosophila ventral furrow morphogenesis: a proteomic analysis. Development 131:643-656.

Goshima Y, Nakamura F, Strittmatter P, Strittmatter SM (1995) Collapsin-induced growth cone collapse mediated by an intracellular protein related to UNC-33. Nature 376:509-514.

Guillery R (1971) An abnormal retinogeniculate projection in the albino ferret (Mustela furo). Brain Res 33:482-485.

Guillery R, LaMantia A, Robson J, Huang K (1985) The influence of retinal afferents upon the development of layers in the dorsal lateral geniculate nucleus of mustelids. J Neurosci 5:1370-1379.

Hendry SH, Yoshioka T (1994) A neurochemically distinct third channel in the macaque dorsal lateral geniculate nucleus. Science 264:575-577.

Hendry SH, Reid RC (2000) The koniocellular pathway in primate vision. Annual Review of Neuroscience 23:127-153.

Hensch TK (2005) Critical period plasticity in local cortical circuits. Nat Rev Neurosci 6:877-888.

Herrera E, Marcus R, Li S, Williams SE, Erskine L, Lai E, Mason C (2004) Foxd1 is required for proper formation of the optic chiasm. Development 131:5727-5739.

Herrera E, Brown L, Aruga J, Rachel RA, Dolen G, Mikoshiba K, Brown S, Mason CA (2003) Zic2 patterns binocular vision by specifying the uncrossed retinal projection. Cell 114:545-557.

Herrmann K, Antonini A, Shatz C (1994) Ultrastructural evidence for synaptic interactions between thalamocortical axons and subplate neurons. Eur J Neurosci 6:1729-1742.

Hogan D, Williams RW (1995) Analysis of the retinas and optic nerves of achiasmatic Belgian sheepdogs. J Comp Neurol 352:367-380.

Hogan D, Garraghty P, Williams R (1999) Asymmetric connections, duplicate layers, and a vertically inverted map in the primary visual system. J Neurosci 19:RC38.

Horng S, Kreiman G, Ellsworth C, Page D, Blank M, Millen K, Sur M (2009) Differential gene expression in the developing lateral geniculate nucleus and medial geniculate nucleus reveals novel roles for Zic4 and Foxp2 in visual and auditory pathway development. J Neurosci 29:13672-13683.

Horton J, Hocking D (1996a) Intrinsic variability of ocular dominance column periodicity in normal macaque monkeys. J Neurosci 16:7228-7239.

Horton J, Hocking D (1996b) An adult-like pattern of ocular dominance columns in striate cortex of newborn monkeys prior to visual experience. J Neurosci 16:1791-1807.

Horton J, Hocking D (1997) Timing of the critical period for plasticity of ocular dominance columns in macaque striate cortex. J Neurosci 17:3684-3709.

Huang Z, Kirkwood A, Pizzorusso T, Porciatti V, Morales B, Bear M, Maffei L, Tonegawa S (1999) BDNF regulates the maturation of inhibition and the critical period of plasticity in mouse visual cortex. Cell 98:739-755.

Hubel D, Wiesel T (1963a) Shape and arrangement of columns in cat's striate cortex. J Physiol 165:559-568.

Hubel D, Wiesel T (1968) Receptive fields and functional architecture of monkey striate cortex. J Physiol (Lond) 195:215-243.

Hubel D, Wiesel T (1969) Anatomical demonstration of columns in the monkey striate cortex. Nature 221:747-750.

Hubel D, Wiesel T (1977a) Ferrier lecture. Functional architecture of macaque monkey visual cortex. Proc R Soc Lond B Biol Sci 198:1-59.

Hubel D, Wiesel T (1998) Early exploration of the visual cortex. Neuron 20:401-412.

Hubel D, Wiesel T, LeVay S (1977) Plasticity of ocular dominance columns in monkey striate cortex. Philos Trans R Soc Lond B Biol Sci 278:377-409.

Hubel DH (1975) An Autoradiographic Study of the Retinal-Cortical Projections in the Tree Shrew (Tupaia glis). Brain Research 96:41-60.

Hubel DH (1982) Exploration of the Primary Visual Cortex, 1955-78. Nature 299:515-524.

Hubel DH, Wiesel TN (1959) Receptive fields of single neurones in the cat's striate cortex. J Physiol 148:574-591.

Hubel DH, Wiesel TN (1962) Receptive Fields, Binocular Interaction and Functional Architecture in the Cat's Visual Cortex. Journal of Physiology 160:106-154.

Hubel DH, Wiesel TN (1963b) Receptive Fields of Cells in Striate Cortex of Very Young, Visually Inexperienced Kittens. Journal of Neurophysiology 26:994-1002.

Hubel DH, Wiesel TN (1966) Effects of varying stimulus size and color on single lateral geniculate cells in Rhesus monkeys. Proc Natl Acad Sci U S A 55:1345-1346.

Hubel DH, Wiesel TN (1970) The Period of Susceptability to the Physiological Effects of Unilateral Eye Closure in Kittens. Journal of Physiology (Lond) 206:419-436.

Hubel DH, Wiesel TN (1972) Laminar and Columnar Distribution of Geniculo-cortical Fibers in the Macaque Monkey. The Journal of Comparitive Neurology 146:421-450.

Hubel DH, Wiesel TN (1974) Sequence Regularity and Geometry of Orientation Columns in the Monkey Striate Cortex. The Journal of Comparitive Neurology 158:267-294.

Hubel DH, Wiesel TN (1977b) Ferrier Lecture: Functional Architecture of the Macaque Monkey Visual Cortex. Proceedings of the Royal Society of London, Series B 198:1-59.

Hubener M (2003) Mouse visual cortex. Curr Opin Neurobiol 13:413-420.

Huberman A, Stellwagen D, Chapman B (2002) Decoupling eye-specific segregation from lamination in the lateral geniculate nucleus. J Neurosci 22:9419-9429.

Huberman A, Murray K, Warland D, Feldheim D, Chapman B (2005) Ephrin-As mediate targeting of eye-specific projections to the lateral geniculate nucleus. Nat Neurosci 8:1013-1021.

Huberman A, Wang G, Liets L, Collins O, Chapman B, Chalupa L (2003) Eye-specific retinogeniculate segregation independent of normal neuronal activity. Science 300:994-998.

Huberman AD, Feller MB, Chapman B (2008) Mechanisms underlying development of visual maps and receptive fields. Annu Rev Neurosci 31:479-509.

Huh G, Boulanger L, Du H, Riquelme P, Brotz T, Shatz C (2000) Functional requirement for class I MHC in CNS development and plasticity. Science 290:2155-2159.

Humphrey A, Albano J, Norton T (1977) Organization of ocular dominance in tree shrew striate cortex. Brain Res 134:225-236.

Hutchins JB, Casagrande VA (1988) Glial cells develop a laminar pattern before neuronal cells in the lateral geniculate nucleus. Proc Natl Acad Sci U S A 85:8316-8320.

Hutchins JB, Casagrande VA (1990) Development of the lateral geniculate nucleus: interactions between retinal afferent, cytoarchitectonic, and glial cell process lamination in ferrets and tree shrews. J Comp Neurol 298:113-128.

Inagaki N, Chihara K, Arimura N, M�nager C, Kawano Y, Matsuo N, Nishimura T, Amano M, Kaibuchi K (2001) CRMP-2 induces axons in cultured hippocampal neurons. Nature Neuroscience 4:781-782.

Issa N, Trachtenberg J, Chapman B, Zahs K, Stryker M (1999) The critical period for ocular dominance plasticity in the Ferret's visual cortex. J Neurosci 19:6965-6978.

Jackson C, Peduzzi J, Hickey T (1989) Visual cortex development in the ferret. I. Genesis and migration of visual cortical neurons. The Journal of Neuroscience 9:1242-1253.

Jacobs S, Van De Plas B, Van Der Gucht E, Clerens S, Cnops L, Van Den Bergh G, Arckens L (2008) Identification of new regional marker proteins to map mouse brain by 2-D difference gel electrophoresis screening. Electrophoresis 29:1518-1524.

Jain E, Bairoch A, Duvaud S, Phan I, Redaschi N, Suzek BE, Martin MJ, McGarvey P, Gasteiger E (2009) Infrastructure for the life sciences: design and implementation of the UniProt website. BMC Bioinformatics 10:136.

Kanold P, Kara P, Reid R, Shatz C (2003) Role of subplate neurons in functional maturation of visual cortical columns. Science 301:521-525.

Kaschube M, Wolf F, Geisel T, Lowel S (2002) Genetic influence on quantitative features of neocortical architecture. J Neurosci 22:7206-7217.

Kaschube M, Schnabel M, Wolf F, Lowel S (2009) Interareal coordination of columnar architectures during visual cortical development. Proc Natl Acad Sci U S A 106:17205-17210.

Kaschube M, Wolf F, Puhlmann M, Rathjen S, Schmidt K, Geisel T, Lowel S (2003) The pattern of ocular dominance columns in cat primary visual cortex: intra- and interindividual variability of column spacing and its dependence on genetic background. Eur J Neurosci 18:3251-3266.

Katz L, Crowley J (2002) Development of cortical circuits: lessons from ocular dominance columns. Nat Rev Neurosci 3:34-42.

Kawasaki H (2004) Molecular Organization of the Ferret Visual Thalamus. Journal of Neuroscience 24:9962-9970.

Kawasaki H, Crowley JC, Livesey FJ, Katz LC (2004) Molecular organization of the ferret visual thalamus. J Neurosci 24:9962-9970.

Keil W, Schmidt KF, Lowel S, Kaschube M (2010) Reorganization of columnar architecture in the growing visual cortex. Proc Natl Acad Sci U S A 107:12293-12298.

Kerschensteiner D, Wong RO (2008) A precisely timed asynchronous pattern of ON and OFF retinal ganglion cell activity during propagation of retinal waves. Neuron 58:851-858.

Kim U, Bal T, McCormick D (1995) Spindle waves are propagating synchronized oscillations in the ferret LGNd in vitro. J Neurophysiol 74:1301-1323.

Kolodkin AL, Levengood DV, Rowe EG, Tai YT, Giger RJ, Ginty DD (1997) Neuropilin is a semaphorin III receptor. Cell 90:753-762.

Kottis V, Thibault P, Mikol D, Xiao ZC, Zhang R, Dergham P, Braun PE (2002) Oligodendrocyte-myelin glycoprotein (OMgp) is an inhibitor of neurite outgrowth. J Neurochem 82:1566-1569.

Krishna K, Nuernberger M, Weth F, Redies C (2009) Layer-specific expression of multiple cadherins in the developing visual cortex (V1) of the ferret. Cereb Cortex 19:388-401.

Law M, Zahs K, Stryker M (1988) Organization of primary visual cortex (area 17) in the ferret. J Comp Neurol 278:157-180.

Leamey CA, Merlin S, Lattouf P, Sawatari A, Zhou X, Demel N, Glendining KA, Oohashi T, Sur M, F�ssler R (2007) Ten-m3 Regulates Eye-Specific Patterning in the Mammalian Visual Pathway and Is Required for Binocular Vision. Plos Biol 5:e241.

Lee K, McCormick D (1997) Modulation of spindle oscillations by acetylcholine, cholecystokinin and 1S,3R-ACPD in the ferret lateral geniculate and perigeniculate nuclei in vitro. Neuroscience 77:335-350.

Lein E, Shatz C (2000) Rapid regulation of brain-derived neurotrophic factor mRNA within eye- specific circuits during ocular dominance column formation. J Neurosci 20:1470-1483.

Lein E, Hohn A, Shatz C (2000) Dynamic regulation of BDNF and NT-3 expression during visual system development. J Comp Neurol 420:1-18.

LeVay S, Ferster D (1977) Relay cell classes in the lateral geniculate nucleus of the cat and the effects of visual deprivation. J Comp Neurol 172:563-584.

LeVay S, Stryker MP, Shatz CJ (1978) Ocular Dominance Columns and Their Development in Layer IV of the Cat's Visual Cortex: A Quantitative Study. Journal of Comparative Neurology 179:223-244.

LeVay S, Wiesel TN, Hubel D (1980) The Development of Ocular Dominance Columns in Normal and Visually Deprived Monkeys. The Journal of Comparative Neurology 191:1-51.

LeVay S, McConnell S, Luskin M (1987) Functional organization of primary visual cortex in the mink (Mustela vison), and a comparison with the cat. J Comp Neurol 257:422-441.

LeVay S, Connolly M, Houde J, Van Essen D (1985) The complete pattern of ocular dominance stripes in the striate cortex and visual field of the macaque monkey. J Neurosci 5:486-501.

Linden D, Guillery R, Cucchiaro J (1981a) The dorsal lateral geniculate nucleus of the normal ferret and its postnatal development. J Comp Neurol 203:189-211.

Linden DC, Guillery RW, Cucchiaro J (1981b) The dorsal lateral geniculate nucleus of the normal ferret and its postnatal development. The Journal of Comparative Neurology 203:189-211.

Livingstone MS, Hubel DH (1984) Anatomy and Physiology of a Color System in the Primate Visual Cortex. The Journal of Neuroscience 4:309-356.

Lyckman AW, Horng S, Leamey CA, Tropea D, Watakabe A, Van Wart A, McCurry C, Yamamori T, Sur M (2008) Gene expression patterns in visual cortex during the critical period: synaptic stabilization and reversal by visual deprivation. Proceedings of the National Academy of Sciences of the United States of America 105:9409-9414.

Mandemakers WJ, Barres BA (2005) Axon regeneration: it's getting crowded at the gates of TROY. Curr Biol 15:R302-305.

McConnell S, LeVay S (1986) Anatomical organization of the visual system of the mink, Mustela vison. J Comp Neurol 250:109-132.

McCormick D, Trent F, Ramoa A (1995) Postnatal development of synchronized network oscillations in the ferret dorsal lateral geniculate and perigeniculate nuclei. The Journal of Neuroscience 15:5739-5752.

McKerracher L, David S, Jackson DL, Kottis V, Dunn RJ, Braun PE (1994) Identification of myelin-associated glycoprotein as a major myelin-derived inhibitor of neurite growth. Neuron 13:805-811.

Meister M, Wong RO, Baylor DA, Shatz CJ (1991) Synchronous bursts of action potentials in ganglion cells of the developing mammalian retina. Science 252:939-943.

Miyashita-Lin EM, Hevner R, Wasserman KM, Martinez S, Rubenstein JLR (1999) Early Neocortical Regionalization in the Absence of Thalamic Innervation. Science 285:906-909.

Morgan JE, Henderson Z, Thompson ID (1987) Retinal decussation patterns in pigmented and albino ferrets. Neuroscience 20:519-535.

Mountcastle V (1957) Modality and topographic properties of single neurons of cat's somatic sensory cortex. J Neurophysiol 20:408-434.

Mountcastle V, Davies P, Berman A (1957) Response properties of neurons of cat's somatic sensory cortex to peripheral stimuli. J Neurophysiol 20:374-407.

Murray KD, Rubin CM, Jones EG, Chalupa LM (2008) Molecular correlates of laminar differences in the macaque dorsal lateral geniculate nucleus. J Neurosci 28:12010-12022.

Murre JM, Sturdy DP (1995) The connectivity of the brain: multi-level quantitative analysis. Biol Cybern 73:529-545.

Nakamura F, Kalb RG, Strittmatter SM (2000) Molecular basis of semaphorin-mediated axon guidance. Journal of Neurobiology 44:219-229.

Nauman JV, Campbell PG, Lanni F, Anderson JL (2007) Diffusion of insulin-like growth factor-I and ribonuclease through fibrin gels. Biophys J 92:4444-4450.

Padmanabhan K (2008) On the role of spontaneous retinal activity in the anatomical development of early visual circuits (Doctoral Dissertaion). Carnegie Mellon University, Pittsburgh, Pennsylvania.

Padmanabhan K, Eddy WF, Crowley JC (2010) A novel algorithm for optimal image thresholding of biological data. J Neurosci Methods 193:380-384.

Pak W, Hindges R, Lim Y-S, Pfaff SL, O'Leary DDM (2004) Magnitude of binocular vision controlled by islet-2 repression of a genetic program that specifies laterality of retinal axon pathfinding. Cell 119:567-578.

Pang ZP, Sudhof TC (2010) Cell biology of Ca2+-triggered exocytosis. Curr Opin Cell Biol 22:496-505.

Penn A, Riquelme P, Feller M, Shatz C (1998a) Competition in Retinogeniculate Patterning Driven by Spontaneous Activity. ScienceNew Series 279:2108-2112.

Penn AA, Riquelme PA, Feller MB, Shatz CJ (1998b) Competition in retinogeniculate patterning driven by spontaneous activity. Science 279:2108-2112.

Perkins D, Pappin D, Creasy D, Cottrell J (1999) Probability-based protein identification by searching sequence databases using mass spectrometry data. Electrophoresis 20:3551-3567.

Pfeiffenberger C, Yamada J, Feldheim D (2006) Ephrin-As and patterned retinal activity act together in the development of topographic maps in the primary visual system. J Neurosci 26:12873-12884.

Pfeiffenberger C, Cutforth T, Woods G, Yamada J, Renteria R, Copenhagen D, Flanagan J, Feldheim D (2005) Ephrin-As and neural activity are required for eye-specific patterning during retinogeniculate mapping. Nat Neurosci 8:1022-1027.

Pistorio AL, Hendry SH, Wang X (2006) A modified technique for high-resolution staining of myelin. Journal of Neuroscience Methods 153:135-146.

Rais I, Karas M, Sch�gger H (2004) Two-dimensional electrophoresis for the isolation of integral membrane proteins and mass spectrometric identification. Proteomics 4:2567-2571.

Rakic P (1981) Development of Visual Centers in the Primate Brain Depends on Binocular Competition Before Birth. Science 214:928-931.

Raper JA (2000) Semaphorins and their receptors in vertebrates and invertebrates. Current Opinion in Neurobiology 10:88-94.

Redies C, Diksic M, Riml H (1990) Functional organization in the ferret visual cortex: a double-label 2-deoxyglucose study. The Journal of Neuroscience 10:2791-2803.

Rockland K (1985) Anatomical organization of primary visual cortex (area 17) in the ferret. J Comp Neurol 241:225-236.

Rowell JJ, Mallik AK, Dugas-Ford J, Ragsdale CW (2010) Molecular analysis of neocortical layer structure in the ferret. J Comp Neurol 518:3272-3289.

Rubenstein JL, Anderson S, Shi L, Miyashita-Lin E, Bulfone A, Hevner R (1999) Genetic control of cortical regionalization and connectivity. Cereb Cortex 9:524-532.

Rusoff A, Dubin M (1977) Development of receptive-field properties of retinal ganglion cells in kittens. J Neurophysiol 40:1188-1198.

Ruthazer E, Stryker M (1996) The role of activity in the development of long-range horizontal connections in area 17 of the ferret. J Neurosci 16:7253-7269.

Schmidt EF, Strittmatter SM (2007) The CRMP family of proteins and their role in Sema3A signaling. Adv Exp Med Biol 600:1-11.

Seung HS (2009) Reading the book of memory: sparse sampling versus dense mapping of connectomes. Neuron 62:17-29.
Shatz C, Luskin M (1986) The relationship between the geniculocortical afferents and their cortical target cells during development of the cat's primary visual cortex. J Neurosci 6:3655-3668.

Shatz CJ (1983) The prenatal development of the cat's retinogeniculate pathway. J Neurosci 3:482-499.

Shatz CJ, Kirkwood PA (1984) Prenatal Development of Functional Connections in the Cat's Retinogeniculate Pathway. The Journal of Neuroscience 4:1378-1397.

Shatz CJ, Stryker MP (1988) Prenatal Tetrodotoxin Infusion Blocks the Segregation of Retinogeniculate Afferents. Science 242:87-89.

Snider RS, Lee JC (1961) A Stereotaxic Atlas of the Monkey Brain. Chicago and London, The University of Chicago Press.

Spatz WB (1979) The retino-geniculo-cortical pathway in Callithrix. II. The geniculo-cortical projection. Exp Brain Res 36:401-410.

Spatz WB (1989) Loss of ocular dominance columns with maturity in the monkey, Callithrix jacchus. Brain Res 488:376-380.

Sperry R (1963) Chemoaffinity in the Orderly Growth of Nerve Fiber Patterns and Connections. Proc Natl Acad Sci U S A 50:703-710.

Sporns O (2011) The human connectome: a complex network. Ann N Y Acad Sci.

Sporns O, Tononi G, Kotter R (2005) The human connectome: A structural description of the human brain. PLoS Comput Biol 1:e42.

Sretavan DW, Shatz CJ (1986a) Prenatal Development of Retinal Ganglion Cell Axons: Segregation into Eye-Specific Layers Within the Cat's Lateral Geniculate Nucleus. The Journal of Neuroscience 6:234-251.

Sretavan DW, Shatz CJ (1986b) Prenatal development of cat retinogeniculate axon arbors in the absence of binocular interactions. J Neurosci 6:990-1003.

Stacy R, Wong R (2003) Developmental relationship between cholinergic amacrine cell processes and ganglion cell dendrites of the mouse retina. J Comp Neurol 456:154-166.

Stellwagen D, Shatz C (2002) An instructive role for retinal waves in the development of retinogeniculate connectivity. Neuron 33:357-367.

Stellwagen D, Shatz C, Feller M (1999) Dynamics of retinal waves are controlled by cyclic AMP. Neuron 24:673-685.

Stryker M (1978) Postnatal development of ocular dominance columns in layer IV of the cat's visual cortex and the effects of monocular deprivation. Arch Ital Biol 116:420-426.

Stryker M, Zahs K (1983) On and off sublaminae in the lateral geniculate nucleus of the ferret. J Neurosci 3:1943-1951.

Stryker M, Harris W (1986) Binocular impulse blockade prevents the formation of ocular dominance columns in cat visual cortex. The Journal of Neuroscience 6:2117-2133.

Thompson ID, Morgan JE (1993) The development of retinal ganglion cell decussation patterns in postnatal pigmented and albino ferrets. Eur J Neurosci 5:341-356.

Tian NM, Pratt T, Price DJ (2008) Foxg1 regulates retinal axon pathfinding by repressing an ipsilateral program in nasal retina and by causing optic chiasm cells to exert a net axonal growth-promoting activity. Development 135:4081-4089.

tom Dieck S, Altrock WD, Kessels MM, Qualmann B, Regus H, Brauner D, Fejtova A, Bracko O, Gundelfinger ED, Brandstatter JH (2005) Molecular dissection of the photoreceptor ribbon synapse: physical interaction of Bassoon and RIBEYE is essential for the assembly of the ribbon complex. J Cell Biol 168:825-836.
Torborg CL, Feller MB (2004) Unbiased analysis of bulk axonal segregation patterns. Journal of Neuroscience Methods 135:17-26.

Tropea D, Van Wart A, Sur M (2009) Molecular mechanisms of experience-dependent plasticity in visual cortex. Philos Trans R Soc Lond, B, Biol Sci 364:341-355.

Tropea D, Kreiman G, Lyckman A, Mukherjee S, Yu H, Horng S, Sur M (2006) Gene expression changes and molecular pathways mediating activity-dependent plasticity in visual cortex. Nat Neurosci 9:660-668.

Unlu M, Morgan M, Minden J (1997) Difference gel electrophoresis: a single gel method for detecting changes in protein extracts. Electrophoresis 18:2071-2077.

Van den Bergh G, Clerens S, Cnops L, Vandesande F, Arckens L (2003) Fluorescent two-dimensional difference gel electrophoresis and mass spectrometry identify age-related protein expression differences for the primary visual cortex of kitten and adult cat. J Neurochem 85:193-205.

Van den Bergh G, Clerens S, Firestein B, Burnat K, Arckens L (2006a) Development and plasticity-related changes in protein expression patterns in cat visual cortex: a fluorescent two-dimensional difference gel electrophoresis approach. Proteomics 6:3821-3832.

Van Den Bergh G, Clerens S, Firestein BL, Burnat K, Arckens L (2006b) Development and plasticity-related changes in protein expression patterns in cat visual cortex: A fluorescent two-dimensional difference gel electrophoresis approach. Proteomics 6:3821-3832.

Van Essen DC, Newsome WT, Maunsell JH (1984) The visual field representation in striate cortex of the macaque monkey: asymmetries, anisotropies, and individual variability. Vision Res 24:429-448.

Vanderhaeghen P (2007) The ephrins and the eph receptors in neural development.1-38.

Vanderhaeghen P, Polleux F (2004) Developmental mechanisms patterning thalamocortical projections: intrinsic, extrinsic and in between. Trends Neurosci 27:384-391.

Venter JC et al. (2001) The sequence of the human genome. Science 291:1304-1351.

Viswanathan S, �nl� M, Minden JS (2006) Two-dimensional difference gel electrophoresis. Nat Protoc 1:1351-1358.

von Krosigk M, Bal T, McCormick D (1993) Cellular mechanisms of a synchronized oscillation in the thalamus. Science 261:361-364.

Wang LH, Strittmatter SM (1996) A family of rat CRMP genes is differentially expressed in the nervous system. J Neurosci 16:6197-6207.

Weliky M (2000) Correlated Neuronal Activity and Visual Cortical Development. Neuron 27:427-430.

Weliky M, Katz LC (1999) Correlational structure of spontaneous neuronal activity in the developing lateral geniculate nucleus in vivo. Science 285:599-604.

White LE, Bosking WH, Williams SM, Fitzpatrick D (1999a) Maps of central visual space in ferret V1 and V2 lack matching inputs from the two eyes. J Neurosci 19:7089-7099.

White LE, Bosking WH, Williams SM, Fitzpatrick D (1999b) Maps of Central Visual Space in Ferret V1 and V2 Lack Matching Inputs from the Two Eyes. The Journal of Neuroscience 19:7089-7099.

Wiesel TN, Hubel DH (1963a) Single Cell Responses in Striate Cortex of Kittens Deprived of Vision in One Eye. Journal of Neurophysiology 26:1003-1017.

Wiesel TN, Hubel DH (1963b) Effects of Visual Deprivation on Morphology and Physiology of Cells in the Cat's Lateral Geniculate Body. Journal of Neurophysiology 26:978-993.

Wiesel TN, Hubel DH (1965) Extent of Recovery from the Effects of Visual Deprivation in Kittens. Journal of Neurophysiology 28:1060-1072.

Wiesel TN, Hubel DH (1966) Spatial and chromatic interactions in the lateral geniculate body of the rhesus monkey. J Neurophysiol 29:1115-1156.

Williams RW, Hogan D, Garraghty PE (1994) Target recognition and visual maps in the thalamus of achiasmatic dogs. Nature 367:637-639.

Williams SE, Mann F, Erskine L, Sakurai T, Wei S, Rossi DJ, Gale NW, Holt CE, Mason CA, Henkemeyer M (2003) Ephrin-B2 and EphB1 mediate retinal axon divergence at the optic chiasm. Neuron 39:919-935.

Wong R (1999) Retinal waves and visual system development. Annu Rev Neurosci 22:29-47.

Wong RO, Meister M, Shatz CJ (1993) Transient period of correlated bursting activity during development of the mammalian retina. Neuron 11:923-938.

Wong ROL, Oakley DM (1996) Changing Patterns of Spontaneous Bursting Activity of On and Off Retinal Ganglion Cells During Development. Neuron 16:1087-1095.

Woolsey TA, Hanaway J, Gado MH (2003) The Brain Atlas: A Visual Guide to the Human Central Nervous System. Hoboken, NJ, John Wiley and Sons, Inc.

Yiu G, He Z (2003) Signaling mechanisms of the myelin inhibitors of axon regeneration. Curr Opin Neurobiol 13:545-551.

Zahs K, Stryker M (1988) Segregation of ON and OFF afferents to ferret visual cortex. The Journal of Neurophysiology 59:1410-1429.