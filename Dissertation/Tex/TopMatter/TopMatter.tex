\title{\Huge{\textbf{Molecular correlates of visual system development}}}
\author{Corey Joseph Flynn\\
		BS, Animal Behavior, Bucknell University 2005
\\		
\\
\\
\\
Submitted to the the faculty in the\\
Department of Biological Sciences at\\
Carnegie Mellon University\\
\\
\\
\\
In partial fulfillment of the requirements for the degree of\\
Doctor of Philosophy 
}
\date{March 2011}
\maketitle
%\doublespacing

\pagenumbering{roman}
\begin{raggedleft} \Huge{\textbf{Abstract}}\end{raggedleft}\\
\Large{\textbf{Molecular correlates of visual system development}}\\
\textbf{Corey Joseph Flynn, March 2011}\\
The development of neural systems has been shown to be under the influence of both neural activity and molecular paradigms.  In the visual system, many classic studies focused on the impact of activity paradigms on the wiring diagram of the system.  In these studies it became clear that altered activity patterns resulted in altered anatomy at the first stages of the visual system.  At the level of the retina, the lateral geniculate nucleus (LGN), and primary visual cortex (V1) the basic wiring diagram of the system can be manipulated by alteration of activity levels during development.  Specifically, much attention has been focused on the processing of visual information from one eye versus the other in this system known as ocular dominance.  Both the physiological and anatomical representations of ocular dominance have been shown to be susceptible to altered activity during development.
  
Over the last few decades, the study of visual development has begun to draw on the tools of molecular biology as well.  It is now clear that molecular paradigms are also at work in neural development.  The influences of molecular paradigms on neural development work in tandem with those of activity based paradigms.  In the visual system, molecular studies have largely focused on the mechanisms governing patterning of topological projections, areal specification, and plasticity mechanisms.  There have been relatively few studies targeted at the molecular characterization of ocular dominance development.  The main goal of this dissertation is to implement a strategy for characterizing molecular correlates of visual system development with a focus on the development of ocular dominance.
  
In order to characterize molecular correlates of ocular dominance in the developing visual system, I first implemented a pair-wise proteomic screen of samples from known anatomical correlates of ocular dominance at the level of the retina, LGN, and V1.  From these screens, many potential candidates were identified and some of them were further characterized through immunohistochemical methods. One of these candidates, collapsin response mediator protein 4 (CRMP4) was shown to be a novel and developmentally regulated marker for LGN development.  CRMP4 expression is present at the interlaminar zones between eye specific laminae in the developing LGN.  Further, expression of CRMP4 is constrained to a short window of development in which the final structure of the LGN is determined.  I show that CRMP4 expression tracks the morphology of the interlaminar zones of the LGN both in normal animals and in animals with altered retinogeniculate inputs.  This finding leads me to hypothesize that the expression of CRMP4 is linked to the interlaminar zones in during the window of development in which it is expressed.  Finally, I suggest that the mechanism of CRMP4 action in the developing LGN may be downstream of myelin associated inhibitor signaling. In this scheme, CRMP4 mediates myelin dependent outgrowth inhibition of retinal ganglion cell afferents at the interlaminar zones and fine tunes the boundaries of eye specific lamina in the developing LGN.\\
\pagebreak

\begin{raggedleft} \Huge{\textbf{Acknowledgments}}\end{raggedleft}\\
The past six years have been a period of profound personal and intellectual growth in my life.  I have never before experienced life so fully.  My time at Carnegie Mellon has challenged me in ways that I did not think possible and I have become a much better scientist and person because of it.  I try to strive for improvement in all that I do and I truly believe that my peers and mentors have facilitated my improvement at every step along the way.  I am appreciative of those that have influenced me throughout my tenure in this program.

First and foremost, I would like to thank my advisor.  Justin Crowley took a chance on me.  I will be forever indebted to him for seeing a budding systems neuroscientist in a young man more accustomed to �boot-wearing biology� and observational behavior work.  Beyond his patience in teaching new methodology, I would like to thank Justin for teaching me his brand of scientific investigation.  Justin introduced a level of rigor and determination into my work that I am not sure I would find in many other laboratories.  He also encouraged me to tackle large and complex problems that I, and many others, may otherwise have shied away from.  I only hope that I can repay these kindnesses with my future endeavors.

Beyond the influence of Justin, I have been positively impacted by all of those that I have been privileged to share lab space with.  In particular, I have benefitted from the feedback of Krishnan Padmanabhan throughout my graduate training.  For this I will be forever grateful.  I would also like to thank David Whitney and Santosh Chandrasekaran for their contributions and conversations over the years.  I have benefitted from the opportunity to mentor many undergraduate students over the years and I would like to thank them for their dedication, effort, and patience.

I would also like to thank the members of my dissertation committee.  Nathan Urban has always been an excellent sounding board for a new perspective on scientific data and many other things in my life.  Jon Minden$'$s kindness and patience was one of the major ingredients in whatever successes are to be found in this dissertation.  Jon allowed me a great deal of access to his laboratory$'$s staff and resources in order to get my work started at Carnegie Mellon and he has continued to be of great support ever since.  Peter Strick has been an incredibly valuable source of input at all of my committee$'$s meeting. He has been able to shape the direction of my work for the better in every conversation that I have had with him. 

In addition to those at Carnegie Mellon, I would like to thank the people that brought me to the start of my training here.  I have benefitted from outstanding mentors from a very young age and my successes are certainly a result of their efforts as much as my own.  I would like to acknowledge the efforts of Thatcher Shug, Mark Evans, Robert Cook, Owen Floody, and Elizabeth Capaldi.

For support outside of my professional life, I would like to thank Gail Siewiorek.  Gail has been there to share in all of my successes and failures during graduate school and she has been a source of strength through it all.  Gail$'$s family has been especially welcoming during my time in Pittsburgh and I would like to thank them for that. 

Last, I would like to acknowledge my family.  My parents, Don and Susan Flynn, as well as my brother, Micah Flynn, have been a constant source of feedback and encouragement throughout my education in the past decade.  None of what I have done would be possible without their support.  For everything that you have done and continue to do each day, thank you.
\pagebreak

\begin{raggedleft} \Huge{\textbf{List of Abbreviations}}\end{raggedleft}\\
\textbf{2D-DIGE} Two dimensional difference gel electrophoresis\\
\textbf{A} Lamina A of the lateral geniculate nucleus\\
\textbf{A1} Lamina A1 of the lateral geniculate nucleus\\
\textbf{C} Laminae C0-C2 of the lateral geniculate nucleus\\
\textbf{C4RIP} CRMP4-RhoA inhibitor protein\\
\textbf{CamK2} Calcium/calmodulin protein kinase 2\\
\textbf{CaV2.2} N-type presynaptic calcium channel\\
\textbf{CRMP2} Collapsin response mediator protein 2\\
\textbf{CRMP4} Collapsin response mediator protein 4\\
\textbf{E\_\_} Embryonic day \_\_\\
\textbf{GAPDH} Glyceraldehyde 3-phosphate dehydrogenase\\
${\bf GSK3\beta}$ Glycogen synthase kinase 3$\beta$\\
\textbf{IEF} Isoelectric focusing\\
\textbf{IPL} Inner plexiform layer\\
\textbf{Isl2} Islet-2\\
\textbf{LGN} Lateral geniculate nucleus\\
\textbf{MALDI-TOF MS} Matrix assisted laser desorbtion/ionization time of flight mass spectrometry\\
\textbf{MAI} Myelin associated ihibitor\\
\textbf{NgR} Nogo receptor\\
\textbf{NSD} No significant difference\\
\textbf{ODC} Ocular dominance column\\
\textbf{ONH} Optic nerve head\\
\textbf{OPL} Outer plexiform layer\\
\textbf{P\_\_} Postnatal day \_\_\\
\textbf{PGN} Perigeniculate nucleus\\
\textbf{RGC} Retinal ganglion cell\\
\textbf{ROCK} Rho kinase\\
\textbf{S1} Primary somatosensory cortex\\
\textbf{TTX} Tetrodotoxin\\
\textbf{V1} Primary visual cortex\\
\textbf{V2} Secondary visual cortex\\

%\doublespacing
\pagebreak

\tableofcontents
\pagebreak

\listoffigures
\pagebreak

\listoftables
\pagebreak

\pagenumbering{arabic}

