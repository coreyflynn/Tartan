\documentclass[12pt]{report}
\usepackage{setspace}
\usepackage{graphicx}
\usepackage[top=1in, bottom=1in, left=1.5in, right=1in]{geometry}
\usepackage{float}
\usepackage{gensymb}
\usepackage[acronym,xindy,nonumberlist]{glossaries}
\usepackage[colorlinks=true,linkcolor=black]{hyperref}
\usepackage{booktabs}
\usepackage{array}
\usepackage{longtable}
\usepackage[table]{xcolor}
\usepackage[size=normalsize]{caption}
\usepackage{indentfirst}
\makeglossaries
%\includeonly{TopMatter/TopMatter}

\begin{document}
\normalsize

\title{\Huge{\textbf{Molecular correlates of visual system development}}}
\author{Corey Joseph Flynn\\
		BS, Animal Behavior, Bucknell University 2005
\\		
\\
\\
\\
Submitted to the the faculty in the\\
Department of Biological Sciences at\\
Carnegie Mellon University\\
\\
\\
\\
In partial fulfillment of the requirements for the degree of\\
Doctor of Philosophy 
}
\date{March 2011}
\maketitle
\doublespacing

\pagenumbering{roman}
\begin{raggedleft} \Huge{\textbf{Abstract}}\end{raggedleft}\\
\Large{\textbf{Molecular correlates of visual system development}}\\
\textbf{Corey Joseph Flynn, March 2011}\\
The development of neural systems has been shown to be under the influence of both neural activity and molecular paradigms.  In the visual system, many classic studies focused on the impact of activity paradigms on the wiring diagram of the system.  In these studies it became clear that altered activity patterns resulted in altered anatomy at the first stages of the visual system.  At the level of the retina, the lateral geniculate nucleus (LGN), and primary visual cortex (V1) the basic wiring diagram of the system can be manipulated by alteration of activity levels during development.  Specifically, much attention has been focused on the processing of visual information from one eye versus the other in this system known as ocular dominance.  Both the physiological and anatomical representations of ocular dominance have been shown to be susceptible to altered activity during development.
  
Over the last few decades, the study of visual development has begun to draw on the tools of molecular biology as well.  It is now clear that molecular paradigms are also at work in neural development.  The influences of molecular paradigms on neural development work in tandem with those of activity based paradigms.  In the visual system, molecular studies have largely focused on the mechanisms governing patterning of topological projections, areal specification, and plasticity mechanisms.  There have been relatively few studies targeted at the molecular characterization of ocular dominance development.  The main goal of this dissertation is to implement a strategy for characterizing molecular correlates of visual system development with a focus on the development of ocular dominance.
  
In order to characterize molecular correlates of ocular dominance in the developing visual system, I first implemented a pair-wise proteomic screen of samples from known anatomical correlates of ocular dominance at the level of the retina, LGN, and V1.  From these screens, many potential candidates were identified and some of them were further characterized through immunohistochemical methods. One of these candidates, collapsin response mediator protein 4 (CRMP4) was shown to be a novel and developmentally regulated marker for LGN development.  CRMP4 expression is present at the interlaminar zones between eye specific laminae in the developing LGN.  Further, expression of CRMP4 is constrained to a short window of development in which the final structure of the LGN is determined.  I show that CRMP4 expression tracks the morphology of the interlaminar zones of the LGN both in normal animals and in animals with altered retinogeniculate inputs.  This finding leads me to hypothesize that the expression of CRMP4 is linked to the interlaminar zones in during the window of development in which it is expressed.  Finally, I suggest that the mechanism of CRMP4 action in the developing LGN may be downstream of myelin associated inhibitor signaling. In this scheme, CRMP4 mediates myelin dependent outgrowth inhibition of retinal ganglion cell afferents at the interlaminar zones and fine tunes the boundaries of eye specific lamina in the developing LGN.\\

\end{document}